\documentclass[11pt]{beamer}
\usetheme{Antibes}
\usepackage[utf8]{inputenc}
\usepackage{amsmath}
\usepackage{amsfonts}
\usepackage{amssymb}
\usepackage{multirow}
\author{Moosa Yahyazadeh}
\title{The Constrained Application Protocol (CoAP)}
%\setbeamercovered{transparent} 
%\setbeamertemplate{navigation symbols}{} 
%\logo{} 
\institute{The University of Iowa} 
%\date{} 
\subject{rfc7252} 
\begin{document}

\begin{frame}
\titlepage
\end{frame}
%------------------------------------------------------------------------------
%\begin{frame}
%\tableofcontents
%\end{frame}
%------------------------------------------------------------------------------
\begin{frame}{What is CoAP?}
\begin{itemize}
\item[•] The Constrained Application Protocol (CoAP) is a specialized web transfer protocol for use with constrained nodes and constrained (e.g., low-power, lossy) networks
\item[•] The protocol is designed for machine-to-machine (M2M) applications
\begin{itemize}
\item[•] smart energy
\item[•] building automation
\end{itemize}
\item[•] It provides a request/response interaction model between application endpoints
\begin{itemize}
\item[•] One design goal $\rightarrow$ keep message overhead small
\begin{itemize}
\item[•] Why? limiting the need for fragmentation in constrained environments
\end{itemize}
\end{itemize}
\end{itemize}
\end{frame}
%------------------------------------------------------------------------------
\begin{frame}{Request/response interaction model}
\begin{itemize}
\item[•] Interaction model of CoAP is similar to the client/server model of HTTP
\item[•] Machine-to-Machine interaction result in CoAP implementation acting in both client and server roles
\item[•] CoAP request/response
\begin{itemize}
\item[•] a request is sent by a client for an action (using a Method Code) on a resource (identified by a URI) on a server
\item[•] server then sends a response with a Response Code; this response may include a resource representation
\end{itemize}
\item[•] Unlike HTTP, CoAP deals with these interchanges asynchronously over a datagram-oriented transport such as UDP.
\begin{itemize}
\item[•] using messages layer that supports optional reliability
\end{itemize}
\end{itemize}
\end{frame}
%------------------------------------------------------------------------------
\begin{frame}
\begin{itemize}
\item[•] Abstract Layering of CoAP
\end{itemize}
\begin{center}
  \begin{tabular}{ | c | c }
    \cline{1-0}
    Application &  \\ \cline{1-0} \hline
    Requests/Responses & \multirow{2}{*}{CoAP}  \\ \cline{1-0} 
    Messages &  \\ \cline{1-0} \hline
    UDP &  \\
    \cline{1-0}
  \end{tabular}
\end{center}
\end{frame}
%------------------------------------------------------------------------------
\end{document}