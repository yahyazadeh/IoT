\documentclass[11pt]{beamer}
\usetheme{Antibes}
\usepackage[utf8]{inputenc}
\usepackage{amsmath}
\usepackage{amsfonts}
\usepackage{amssymb}
\usepackage{multirow}
\author{Moosa Yahyazadeh}
\title{The Constrained Application Protocol (CoAP)}
%\setbeamercovered{transparent} 
%\setbeamertemplate{navigation symbols}{} 
%\logo{} 
\institute{The University of Iowa} 
%\date{} 
\subject{rfc7252} 
\begin{document}

\begin{frame}
\titlepage
\end{frame}
%------------------------------------------------------------------------------
%\begin{frame}
%\tableofcontents
%\end{frame}
%------------------------------------------------------------------------------
\begin{frame}{What is CoAP?}
\begin{itemize}
\item[•] The Constrained Application Protocol (CoAP) is a specialized web transfer protocol for use with constrained nodes and constrained (e.g., low-power, lossy) networks
\item[•] The protocol is designed for machine-to-machine (M2M) applications
\begin{itemize}
\item[•] smart energy
\item[•] building automation
\end{itemize}
\item[•] It provides a request/response interaction model between application endpoints
\begin{itemize}
\item[•] One design goal $\rightarrow$ keep message overhead small
\begin{itemize}
\item[•] Why? limiting the need for fragmentation in constrained environments
\end{itemize}
\end{itemize}
\end{itemize}
\end{frame}
%------------------------------------------------------------------------------
\begin{frame}{Request/response interaction model}
\begin{itemize}
\item[•] Interaction model of CoAP is similar to the client/server model of HTTP
\item[•] Machine-to-Machine interaction result in CoAP implementation acting in both client and server roles
\item[•] CoAP request/response
\begin{itemize}
\item[•] a request is sent by a client for an action (using a Method Code) on a resource (identified by a URI) on a server
\item[•] server then sends a response with a Response Code; this response may include a resource representation
\end{itemize}
\item[•] Unlike HTTP, CoAP deals with these interchanges asynchronously over a datagram-oriented transport such as UDP.
\begin{itemize}
\item[•] using messages layer that supports optional reliability
\end{itemize}
\end{itemize}
\end{frame}
%------------------------------------------------------------------------------
\begin{frame}{CoAP layers}
\begin{itemize}
\item[•] Abstract Layering of CoAP
\end{itemize}
\begin{center}
  \begin{tabular}{ | c | c }
    \cline{1-0}
    Application &  \\ \cline{1-0} \hline
    Requests/Responses & \multirow{2}{*}{CoAP}  \\ \cline{1-0} 
    Messages &  \\ \cline{1-0} \hline
    UDP &  \\
    \cline{1-0}
  \end{tabular}
\end{center}
\begin{itemize}
\item[•] One could think of CoAP logically as using a two-layer approach, a CoAP messaging layer used to deal with UDP and the asynchronous nature of the interactions, and the request/response interactions using Method and Response Codes
\item[•] CoAP is however a single protocol, with messaging and request/response as just features
   of the CoAP header.
\end{itemize}
\end{frame}
%------------------------------------------------------------------------------
\begin{frame}{Messaging model}
\begin{itemize}
\item[•] Message types
\begin{itemize}
\item[•] Confirmable
\item[•] Non-confirmable
\item[•] Acknowledgement
\item[•] Reset
\end{itemize}
\item[•] Method Codes and Response Codes included in some of these messages make them carry requests or responses
\item[•] The basic exchanges of the four types of messages are somewhat orthogonal to the request/response interactions; requests can be carried in Confirmable and Non-confirmable messages, and responses can be carried in these as well as piggybacked in Acknowledgement messages
\begin{itemize}
\item[•] Thus, Requests cannot be carried in Ack messages
\end{itemize}
\end{itemize}
\end{frame}
%------------------------------------------------------------------------------
\begin{frame}{Messaging model (Cont...)}
\begin{itemize}
\item[•] Each message contains a Message ID 
\item[•] Reliability is provided by marking a message as Confirmable (CON)
\begin{itemize}
\item[•] A Confirmable message is retransmitted using a default timeout and exponential back-off between retransmissions, until the recipient sends an Acknowledgement message (ACK) with the same Message ID from the corresponding endpoint
\item[•] When a recipient is not at all able to process a Confirmable message (i.e., not even able to provide a suitable error response), it \textbf{replies} with a Reset message (RST) instead of an Acknowledgement (ACK).
\end{itemize}
\item[•] Non-confirmable messages are not acknowledged, but still have a Message ID for duplicate detection
\begin{itemize}
\item[•] When a recipient is not able to process a Non-confirmable message, it \textbf{may reply} with a Reset message (RST)
\end{itemize}
\end{itemize}
\end{frame}
%------------------------------------------------------------------------------
\begin{frame}{Request/response model}
\begin{itemize}
\item[•] Request and response semantics are carried in CoAP messages
\begin{itemize}
\item[•] Request $\rightarrow$ include Method Code
\item[•] Response $\rightarrow$ include Response Code
\item[•] Optional (or default) request and response information, such as the URI and payload media type are carried as CoAP Options
\item[•] matching responses to requests independently from the underlying messages is done by Token
\begin{itemize}
\item[•] Notice: Token is a concept separate from the Message ID
\end{itemize}
\end{itemize}
\item[•] Scenario: A request is carried in a Confirmable (CON) 
\begin{itemize}
\item[•] If immediately available, the response to a request carried in a Confirmable message is carried in the resulting (ACK) message
\begin{itemize}
\item[•] Called a piggybacked response; same Msg ID, same Token
\item[•] No need for separately acknowledging a piggybacked response, as the client will retransmit the request if the (ACK) message carrying the piggybacked response is lost
\end{itemize}
\end{itemize}
\end{itemize}
\end{frame}
%------------------------------------------------------------------------------
\begin{frame}{Request/response model (Cont...)}
\begin{itemize}
\item[•] Scenario: A request is carried in a (CON) and the server is not able to respond immediately (with Pigg. Res.)
\begin{itemize}
\item[•] It simply responds with an Empty Ack message so that the client can stop retransmitting the request
\item[•] When the response is ready, the server sends it in a new Confirmable message (which then in turn needs to be acknowledged by the client)
\begin{itemize}
\item[•] called a "separate response"
\item[•] different Msg ID, same Token
\end{itemize}
\end{itemize}
\item[•] Scenario: A request is carried in a Non-confirmable (NON)
\begin{itemize}
\item[•] the response is sent using a new Non-confirmable message, although the server may instead send a Confirmable message
\begin{itemize}
\item[•] The response $\rightarrow$ different Msg ID, same Token
\end{itemize}
\end{itemize}
\end{itemize}
\end{frame}
%------------------------------------------------------------------------------
\begin{frame}{Request/response model (Cont...)}
\begin{itemize}
\item[•] Methods in CoAP:
\begin{itemize}
\item[•] GET, PUT, POST, and DELETE
\begin{itemize}
\item[•] Similar manner to HTTP, but there are some differences
\end{itemize}
\item[•] Methods beyond the basic four can be added to CoAP in separate specifications
\begin{itemize}
\item[•] New methods do not necessarily have to use requests and responses in pairs
\item[•] Even for existing methods, a single request may yield multiple responses, e.g., for a multicast request or with the Observe option
\end{itemize}
\end{itemize}
\end{itemize}
\end{frame}
%------------------------------------------------------------------------------
\begin{frame}{Request/Response Semantics}
\begin{itemize}
\item[•] CoAP operates under a similar request/response model as HTTP
\begin{itemize}
\item[•] a CoAP endpoint in the role of a "client" sends one or more CoAP requests to a "server", which services the requests by sending CoAP responses
\end{itemize}
\item[•] Unlike HTTP, requests and responses are not sent over a previously established connection but are exchanged \textbf{asynchronously} over CoAP messages
\end{itemize}
\end{frame}
%------------------------------------------------------------------------------
\begin{frame}{Request/Response Semantics (Cont...)}
\begin{itemize}
\item[•] Request
\begin{itemize}
\item[•] Contains
\begin{itemize}
\item[•] a method to be applied to the resource
\item[•] the identifier of the resource
\item[•] a payload and Internet media type (if any)
\item[•] and optional metadata about the request
\end{itemize}
\end{itemize}
\item[•] Response
\begin{itemize}
\item[•] After receiving and interpreting a request, a server responds with a CoAP response that is matched to the request by means of a client-generated token; note that this is different from the Message ID that matches a Confirmable message to its Acknowledgement.
\item[•] can be sent as Piggybacked, Separate 
\end{itemize}
\end{itemize}
\end{frame}
%------------------------------------------------------------------------------
\begin{frame}{Request/Response Semantics (Cont...)}
\begin{itemize}
\item[•] Piggybacked response
\begin{itemize}
\item[•] Response (whether success or failure) is carried directly in the Ack message (request was carried in a Confirmable message)
\item[•] The protocol leaves the decision whether to piggyback a response or not to the server
\item[•] The client MUST be prepared to receive either
\item[•] There is a strong expectation that servers will use piggyback whenever possible
\begin{itemize}
\item[•] Saving resources in the network
\end{itemize}
\end{itemize}
\end{itemize}
\end{frame}
%------------------------------------------------------------------------------
\begin{frame}{Request/Response Semantics (Cont...)}
\begin{itemize}
\item[•] Separate response
\begin{itemize}
\item[•] Whenever that is not possible to return a piggybacked response
\begin{itemize}
\item[•] Server might need longer to obtain the representation of the resource requested than it can wait to send back the Ack message (Using a Ack timer), without risking the client repeatedly retransmitting the request message
\item[•] The response to a request carried in a Non-confirmable message is always sent separately
\end{itemize}
\item[•] The sep. res. can be Confirmable or not (server's decision)
\item[•] Implementation Note:  The protocol leaves the decision whether to piggyback a response or not (i.e., send a separate response) to the server. If possible piggyback is better. The client MUST be prepared to receive either. 
\end{itemize}
\end{itemize}
\end{frame}
%------------------------------------------------------------------------------
\begin{frame}{Request/Response Semantics (Cont...)}
\begin{itemize}
\item[•] Separate response
\begin{itemize}
\item[•] Logic
\begin{itemize}
\item[•] It sends the Ack to the Confirmable req as an Empty message
\item[•] Once it's done, server must not send back the res. in another Ack, even if the client retransmits another identical req.
\item[•] If a retransmitted request is received (perhaps because the original Ack was delayed), another Empty Ack is sent, and any response MUST be sent as a separate response
\item[•] If the server then sends a Confirmable response, the client’s Ack to that response MUST also be an Empty message (one that carries neither a request nor a response)
\item[•] The server MUST stop retransmitting its response on any matching Ack or Reset message
\item[•] If sep. res. arrived before ack (it may happen in UDP), this also serves as an ack 
\item[•] It the res won't come within reasonable time, client should back off (better to to set up a timeout that is unrelated to retransmission timers)
\end{itemize}
\end{itemize}
\end{itemize}
\end{frame}
%------------------------------------------------------------------------------
\begin{frame}{Message format}
\begin{itemize}
\item[•] CoAP is based on the exchange of compact messages
\begin{itemize}
\item[•] transported over UDP $\rightarrow$ each CoAP message occupies the data section of one UDP datagram
\item[•] may also be used over Datagram Transport Layer Security (DTLS)
\item[•] It could also be used over other transports such as SMS, TCP, or SCTP
\item[•] UDP-lite [RFC3828] and UDP zero checksum [RFC6936] are not supported by CoAP
\end{itemize}
\end{itemize}
\end{frame}
%------------------------------------------------------------------------------
\begin{frame}{Message format (Cont...)}
\begin{itemize}
\item[•] messages are encoded in a simple binary format
\item[•] message structure
\begin{itemize}
\item[•] header
\item[•] token (if any)
\item[•] options (if any)
\item[•] payload marker (if there is any payload)
\item[•] payload (if any)
\end{itemize}
\end{itemize}
\begin{center}
  \begin{tabular}{*{32}{c}}
    \hline
    \multicolumn{2}{|c|}{{\scriptsize Ver}} & \multicolumn{2}{|c|}{\scriptsize T} & \multicolumn{4}{|c|}{\scriptsize TKL} & \multicolumn{8}{|c|}{\scriptsize Code} & \multicolumn{16}{|c|}{\scriptsize Message ID} \\ \hline
    \multicolumn{32}{|l}{\scriptsize Token (if any, TKL bytes) ...} \\ \hline
    \multicolumn{32}{|l}{\scriptsize Options (if any) ...} \\ \hline
    \multicolumn{8}{|c|}{\scriptsize 11111111} & \multicolumn{24}{|l}{\scriptsize Payload (if any) ...} \\ \hline 
  \end{tabular}
\end{center}
\end{frame}
%------------------------------------------------------------------------------
\begin{frame}{Message format - Header}
\begin{itemize}
\item[•] header (fixed-size 4-byte)
\begin{itemize}
\item[•] Version (Ver):  2-bit unsigned integer
\begin{itemize}
\item[•] CoAP version number
\item[•] Implementations of this specification \textbf{MUST} set this field to 1 (01 binary).  Other values are reserved for future versions. Messages with unknown version numbers \textbf{MUST} be silently ignored
\end{itemize}
\item[•] Type (T):  2-bit unsigned integer
\begin{itemize}
\item[•] Indicates if this message is of type Confirmable (0), Non-confirmable (1), Acknowledgement (2), or Reset (3)
\end{itemize}
\item[•] Token Length (TKL):  4-bit unsigned integer
\begin{itemize}
\item[•] Indicates the length of the variable-length Token field \textbf{(0-8 bytes)}
\item[•] Lengths \textbf{9-15} are reserved, \textbf{MUST NOT} be sent, and \textbf{MUST} be processed as a \textbf{message format error}
\end{itemize}
\end{itemize}
\end{itemize}
\end{frame}
%------------------------------------------------------------------------------
\begin{frame}{Message format - Header (Cont...)}
\begin{itemize}
\item[•] header (fixed-size 4-byte) (Cont...)
\begin{itemize}
\item[•] Code:  8-bit unsigned integer
\begin{itemize}
\item[•] split into a 3-bit class (most significant bits) and a 5-bit detail (least significant bits)
\item[•] documented as "c.dd". "c" is a digit from 0 to 7 for the 3-bit subfield and "dd" are two digits from 00 to 31 for the 5-bit subfield. 
\item[•] The class can indicate a request (0), a success response (2), a client error response (4), or a server error response (5).  (All other class values are reserved.)
\item[•] As a special case, Code 0.00 indicates an Empty message.
\item[•] In case of a \textbf{request} (Class=0), the Code field indicates the \textbf{Request Method}
\item[•] In case of a \textbf{response} (Class $\in \left\lbrace  2,4,5 \right\rbrace $), the Code field indicates a \textbf{Response Code}.
\end{itemize}
\end{itemize}
\end{itemize}
\end{frame}
%------------------------------------------------------------------------------
\begin{frame}{Message format - Header (Cont...)}
\begin{itemize}
\item[•] header (fixed-size 4-byte) (Cont...)
\begin{itemize}
\item[•] Message ID:  16-bit unsigned integer
\begin{itemize}
\item[•] Used to detect message duplication
\item[•] Used to to match messages of type Acknowledgement/Reset to messages of type Confirmable/Non-confirmable.
\item[•] These rules will be described later
\item[•] Its 16-bit size enables up to about 250 messages per second from one endpoint to another with default protocol parameters
\end{itemize}
\end{itemize}
\end{itemize}
\end{frame}
%------------------------------------------------------------------------------
\begin{frame}{Message format - Token}
\begin{itemize}
\item[•] Token
\begin{itemize}
\item[•] Token value may be 0 to 8 bytes as given by the Token Length field
\item[•] Used to correlate requests and responses
\item[•] Every message carries a token, even if it is of zero length
\item[•] Every request carries a client-generated token that the server MUST echo (without modification) in any resulting response
\item[•] Intended for use as a client-local identifier for differentiating between concurrent requests; could be called a "request ID"
\item[•] SHOULD use a nontrivial, randomized token to guard against spoofing of responses (if it doesn't use Transport Layer Security)
\item[•] An endpoint receiving a token it did not generate MUST treat the token as opaque and make no assumptions about its content or structure.
\end{itemize}
\end{itemize}
\end{frame}
%------------------------------------------------------------------------------
\begin{frame}{Message format - Options}
\begin{itemize}
\item[•] Options
\begin{itemize}
\item[•] There can be zero or more Options
\item[•] An Option can be followed by the end of the message, by another Option, or by the Payload Marker and the payload
\item[•] Each option instance specifies the Option Number, the length of the Option Value, and Option Value.
\begin{itemize}
\item[•] Option number is calculated using simply summing the Option Delta values of this and all previous options before it
\end{itemize}
\end{itemize}
\end{itemize}
\end{frame}
%------------------------------------------------------------------------------
\begin{frame}{Request/Response Matching Rules}
\begin{itemize}
\item[•] The source endpoint of the response MUST be the same as the destination endpoint of the original request.
\item[•] In a piggybacked response, the Message ID of the Confirmable request and the Acknowledgement MUST match, and the tokens of the response and original request MUST match.  
\item[•] In a separate response, just the tokens of the response and original request MUST match.
\item[•] In case a message carrying a response is unexpected (the client is not waiting for a response from the identified endpoint, at the endpoint addressed, and/or with the given token), the response is rejected
\end{itemize}
\end{frame}
%------------------------------------------------------------------------------
\begin{frame}{Code registries}
\begin{itemize}
\item[•] Code ranges:
\begin{itemize}
\item[•] 0.00 $\rightarrow$ Indicates an Empty message
\item[•] 0.01-0.31 $\rightarrow$ Indicates a request
\begin{itemize}
\item[•] Initial CoAP Method Codes: 0.01 $\rightarrow$ GET, 0.02 $\rightarrow$ POST, 0.03 $\rightarrow$ PUT, 0.04 $\rightarrow$ DELETE
\item[•] All other Method Codes are Unassigned
\end{itemize}
\item[•] 1.00-1.31 $\rightarrow$ Reserved
\item[•] 2.00-5.31 $\rightarrow$ Indicates a response
\begin{itemize}
\item[•] The Response Codes 3.00-3.31 are Reserved for future use
\end{itemize}
\item[•] 6.00-7.31 $\rightarrow$ Reserved
\end{itemize}
\end{itemize}
\end{frame}
%------------------------------------------------------------------------------
\begin{frame}{Code registries (Cont...)}
\begin{itemize}
\item[•] Response codes
\begin{itemize}
\item[•] 2.01 $\rightarrow$ Created, 2.02 $\rightarrow$ Deleted, 2.03 $\rightarrow$ Valid, 2.04 $\rightarrow$ Changed, 2.05 $\rightarrow$ Content
\item[•] 4.00 $\rightarrow$ Bad Request, 4.01 $\rightarrow$ Unauthorized, 4.02 $\rightarrow$ Bad Option, 4.03 $\rightarrow$ Forbidden, 4.04 $\rightarrow$ Not Found, 4.05 $\rightarrow$ Method Not Allowed, 4.06 $\rightarrow$ Not Acceptable, 4.12 $\rightarrow$ Precondition Failed, 4.13 $\rightarrow$ Request Entity Too Large, 4.15 $\rightarrow$ Unsupported Content-Format
\begin{itemize}
\item[•] All other codes of the class should be treated as the generic response code of the class (4.00)
\end{itemize}
\item[•] 5.00 $\rightarrow$ Internal Server Error, 5.01 $\rightarrow$ Not Implemented, 5.02 $\rightarrow$ Bad Gateway, 5.03 $\rightarrow$ Service Unavailable, 5.04 $\rightarrow$ Gateway Timeout, 5.05 $\rightarrow$ Proxying Not Supported
\begin{itemize}
\item[•] All other codes of the class should be treated as the generic response code of the class (5.00)
\end{itemize}
\item[•] The Response Codes 3.00-3.31 are Reserved for future use
\item[•] All other Response Codes are Unassigned
\end{itemize}
\end{itemize}
\end{frame}
%------------------------------------------------------------------------------
\begin{frame}{Multicast CoAP}
Section 8 discusses the proper use of CoAP messages with multicast addresses and precautions for avoiding response congestion
\end{frame}
%------------------------------------------------------------------------------
\end{document}